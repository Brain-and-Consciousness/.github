\documentclass[11pt, oneside]{article}   	% use "amsart" instead of "article" for AMSLaTeX format
\usepackage{geometry}                		% See geometry.pdf to learn the layout options. There are lots.
\geometry{letterpaper}                   		% ... or a4paper or a5paper or ... 
%\geometry{landscape}                		% Activate for rotated page geometry
%\usepackage[parfill]{parskip}    		% Activate to begin paragraphs with an empty line rather than an indent
\usepackage{graphicx}				% Use pdf, png, jpg, or eps§ with pdflatex; use eps in DVI mode
								% TeX will automatically convert eps --> pdf in pdflatex		
\usepackage{amssymb}

%SetFonts

%SetFonts


\title{Towards more universal design in the art of magic}
\author{Vebjørn Ekroll}
\date{2024-11-19}							% Activate to display a given date or no date

\begin{document}
\maketitle

\noindent \textbf{Abstract}: In the art of conjuring, as well as in cognitive science, potential possibilities for designing magic tricks that are suitable for people who are blind or visually impaired have only rarely been considered. In this talk, I will give a brief overview over previous efforts in this direction and discuss potential avenues for progress. Recent work of ours suggests that magic tricks based on systematic illusions in spatial imagery can be successfully adapted for non-visual presentation, and thus provide a promising tool for making the art of magic more inclusive. Much more can be done in this area, though, and I argue that efforts directed towards the aim of making magic accessible to people with blindness can not only make the art more inclusive, but also advance basic research in perception and cognition.

%\subsection{}



\end{document}  